\documentclass[12pt]{article}
\usepackage{pmmeta}
\pmcanonicalname{CubicSplineInterpolation}
\pmcreated{2013-03-22 13:40:25}
\pmmodified{2013-03-22 13:40:25}
\pmowner{yota}{10184}
\pmmodifier{yota}{10184}
\pmtitle{cubic spline interpolation}
\pmrecord{7}{34339}
\pmprivacy{1}
\pmauthor{yota}{10184}
\pmtype{Definition}
\pmcomment{trigger rebuild}
\pmclassification{msc}{65-01}

% this is the default PlanetMath preamble.  as your knowledge
% of TeX increases, you will probably want to edit this, but
% it should be fine as is for beginners.

% almost certainly you want these
\usepackage{amssymb}
\usepackage{amsmath}
\usepackage{amsfonts}

% used for TeXing text within eps files
%\usepackage{psfrag}
% need this for including graphics (\includegraphics)
%\usepackage{graphicx}
% for neatly defining theorems and propositions
%\usepackage{amsthm}
% making logically defined graphics
%%%\usepackage{xypic}

% there are many more packages, add them here as you need them

% define commands here
\begin{document}
Suppose we are given $N+1$ data points $\{(x_{k},y_{k})\}$ such
that

\begin{equation}
a=x_{0}<\dots<x_{N}. \label{knot}
\end{equation}

Then the function $S(x)$ is called a \emph{cubic spline interpolation} if
there exists $N$ cubic polynomials $S_{k}(x)$ with coefficients
$s_{k,i}\,\,0\leq i\leq 3$ such that the following hold.


\begin{enumerate}
\item $S(x)=S_{k}(x)=\sum_{i=0}^{3}s_{k,i}(x-x_{k})^{i} \;  \;\;\forall x\in
[x_{k},x_{k+1}]\;\;\;0\leq k \leq N-1$
\item $S(x_{k})=y_{k}\;\;\;0\leq k \leq N$
\item $S_{k}(x_{k+1})=S_{k+1}(x_{k+1})\;\;\;0\leq k \leq N-2$
\item $S^{\prime}_{k}(x_{k+1})=S^{\prime}_{k+1}(x_{k+1})\;\;\;0\leq k \leq N-2$
\item $S^{\prime \prime}_{k}(x_{k+1})=S^{\prime \prime}_{k+1}(x_{k+1})\;\;\;0\leq k \leq N-2$
\end{enumerate}

The set of points $\eqref{knot}$ are called the knots.  The set of
cubic splines on a fixed set of knots, forms a vector space for
cubic spline addition and scalar multiplication.

So we see that the cubic spline not only interpolates the data
$\{(x_{k},y_{k})\}$ but matches the first and second derivatives
at the knots. Notice, from the above definition, one is free to
specify constraints on the endpoints. One common end point
constraint is $S^{\prime\prime}(a)=0\;\;S^{\prime\prime}(b)=0$,
which is called the natural spline. Other popular choices are the
clamped cubic spline, parabolically terminated spline and
curvature-adjusted spline. Cubic splines are frequently used in
numerical analysis to fit data. Matlab uses the command spline to
find cubic spline interpolations with not-a-knot end point
conditions. For example, the following commands would find the
cubic spline interpolation of the curve $4\cos(x)+1$ and plot the
curve and the interpolation marked with $\text{o's}$.



\begin{verbatim}
x = 0:2*pi;
y = 4*cos(x)+1;
 xx = 0:.001:2*pi;
yy = spline(x,y,xx);
plot(x,y,'o',xx,yy)
\end{verbatim}
%%%%%
%%%%%
\end{document}
