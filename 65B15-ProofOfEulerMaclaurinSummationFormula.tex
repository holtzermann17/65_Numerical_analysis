\documentclass[12pt]{article}
\usepackage{pmmeta}
\pmcanonicalname{ProofOfEulerMaclaurinSummationFormula}
\pmcreated{2013-03-22 13:28:41}
\pmmodified{2013-03-22 13:28:41}
\pmowner{pbruin}{1001}
\pmmodifier{pbruin}{1001}
\pmtitle{proof of Euler-Maclaurin summation formula}
\pmrecord{5}{34048}
\pmprivacy{1}
\pmauthor{pbruin}{1001}
\pmtype{Proof}
\pmcomment{trigger rebuild}
\pmclassification{msc}{65B15}
%\pmkeywords{number theory}
%\pmkeywords{numerical integration}

\endmetadata

% this is the default PlanetMath preamble.  as your knowledge
% of TeX increases, you will probably want to edit this, but
% it should be fine as is for beginners.

% almost certainly you want these
\usepackage{amssymb}
\usepackage{amsmath}
\usepackage{amsfonts}

% used for TeXing text within eps files
%\usepackage{psfrag}
% need this for including graphics (\includegraphics)
%\usepackage{graphicx}
% for neatly defining theorems and propositions
%\usepackage{amsthm}
% making logically defined graphics
%%%\usepackage{xypic}

% there are many more packages, add them here as you need them

% define commands here
\def\dd{\mathrm{d}}
\begin{document}
Let $a$ and $b$ be integers such that $a<b$, and let
$f:[a,b]\to\mathbb{R}$ be continuous.  We will prove by induction that
for all integers $k\ge 0$, if $f$ is a $C^{k+1}$ function,
\begin{equation}
\label{em1}
\sum_{a<n\le b}f(n) = \int_a^b f(t)\dd t + \sum_{r=0}^k
\frac{(-1)^{r+1}B_{r+1}}{(r+1)!}  (f^{(r)}(b)-f^{(r)}(a)) +
\frac{(-1)^k}{(k+1)!}  \int_a^b B_{k+1}(t)f^{(k+1)}(t)\dd t
\end{equation}
where $B_r$ is the $r$th Bernoulli number and $B_r(t)$ is the $r$th
Bernoulli periodic function.

To prove the formula for $k=0$, we first rewrite
$\int_{n-1}^{n}f(t)\dd t$, where $n$ is an integer, using
integration by parts:
\begin{eqnarray*}
\int_{n-1}^n f(t)\dd t&=&\int_{n-1}^n\frac{\dd}{\dd t}
(t-n+\frac{1}{2})f(t)\dd t\\
&=&(t-n+\frac{1}{2})f(t)\big\vert_{n-1}^n -
\int_{n-1}^n (t-n+\frac{1}{2})f'(t)\dd t\\
&=&\frac{1}{2}(f(n)+f(n-1))-\int_{n-1}^n (t-n+\frac{1}{2})f'(t)\dd t.
\end{eqnarray*}
Because $t-n+\frac{1}{2}=B_1(t)$ on the interval $(n-1,n)$, this is
equal to
$$
\int_{n-1}^n f(t)\dd
t=\frac{1}{2}(f(n)+f(n-1))-\int_{n-1}^n B_1(t)f'(t)\dd t.
$$
From this, we get
$$
f(n)=\int_{n-1}^n f(t)\dd t+\frac{1}{2}(f(n)-f(n-1))+
\int_{n-1}^n B_1(t)f'(t)\dd t.
$$
Now we take the sum of this expression for $n=a+1,a+2,\ldots,b$, so
that the middle term on the right telescopes away for the most part:
$$
\sum_{n=a+1}^b f(n)=\int_a^b f(t)\dd t+\frac{1}{2}(f(b)-f(a))+
\int_a^b B_1(t)f'(t)\dd t
$$
which is the Euler-Maclaurin formula for $k=0$, since
$B_1=-\frac{1}{2}$.

Suppose that $k>0$ and the formula is correct for $k-1$, that is
\begin{equation}
\label{em2}
\sum_{a<n\le b}f(n) = \int_a^b f(t)\dd t + \sum_{r=0}^{k-1}
\frac{(-1)^{r+1} B_{r+1}}{(r+1)!} (f^{(r)}(b)-f^{(r)}(a)) +
\frac{(-1)^{k-1}}{k!} \int_a^b B_k(t)f^{(k)}(t)\dd t.
\end{equation}
We rewrite the last integral using integration by parts and the facts
that $B_k$ is continuous for $k\ge 2$ and $B_{k+1}'(t)=(k+1)B_k(t)$
for $k\ge 0$:
\begin{eqnarray*}
\int_a^b B_k(t)f^{(k)}(t)\dd t&=&\int_a^b \frac{B_{k+1}'(t)}{k+1}
f^{(k)}(t) \dd t\\
&=&\frac{1}{k+1}B_{k+1}(t)f^{(k)}(t)\big\vert_a^b - \frac{1}{k+1}
\int_a^b B_{k+1}(t) f^{(k+1)}(t)\dd t.
\end{eqnarray*}
Using the fact that $B_k(n)=B_k$ for every integer $n$ if $k\ge 2$, we
see that the last term in Eq. \ref{em2} is equal to
$$
\frac{(-1)^{k+1}B_{k+1}}{(k+1)!}(f^{(k)}(b)-f^{(k)}(a)) +
\frac{(-1)^k}{(k+1)!} \int_a^b B_{k+1}(t) f^{(k+1)}(t)\dd t.
$$
Substituting this and absorbing the left term into the summation
yields Eq. \ref{em1}, as required.
%%%%%
%%%%%
\end{document}
