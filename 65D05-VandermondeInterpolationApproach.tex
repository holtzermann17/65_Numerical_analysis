\documentclass[12pt]{article}
\usepackage{pmmeta}
\pmcanonicalname{VandermondeInterpolationApproach}
\pmcreated{2013-03-22 14:19:56}
\pmmodified{2013-03-22 14:19:56}
\pmowner{mathcam}{2727}
\pmmodifier{mathcam}{2727}
\pmtitle{Vandermonde interpolation approach}
\pmrecord{8}{35802}
\pmprivacy{1}
\pmauthor{mathcam}{2727}
\pmtype{Definition}
\pmcomment{trigger rebuild}
\pmclassification{msc}{65D05}
\pmclassification{msc}{41A05}

% this is the default PlanetMath preamble.  as your knowledge
% of TeX increases, you will probably want to edit this, but
% it should be fine as is for beginners.

% almost certainly you want these
\usepackage{amssymb}
\usepackage{amsmath}
\usepackage{amsfonts}

% used for TeXing text within eps files
%\usepackage{psfrag}
% need this for including graphics (\includegraphics)
%\usepackage{graphicx}
% for neatly defining theorems and propositions
%\usepackage{amsthm}
% making logically defined graphics
%%%\usepackage{xypic}

% there are many more packages, add them here as you need them

% define commands here
\begin{document}
The Vandermonde approach for interpolation is when we wish to determine the interpolating polynomial $p(x)=a_0 + a_1x + a_2x^2 + \ldots +a_nx^n$ for the $n+1$ points $(x_i,y_i)$, $i=0,1,\ldots, n$ by forming the equations $y_i=a_0+a_1x_i+a_2x_2^2+\ldots+a_nx_n^n$ for $i = 0,1,\ldots,n$, and solving for the unknown coefficients $a_0, a_1,\ldots, a_n$. 

The system of equations can be written by using matrices $Y=XA$ where $X$ is a Vandermonde matrix.
%%%%%
%%%%%
\end{document}
