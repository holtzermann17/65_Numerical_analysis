\documentclass[12pt]{article}
\usepackage{pmmeta}
\pmcanonicalname{PropertiesOfBernsteinPolynomial}
\pmcreated{2013-03-22 17:24:13}
\pmmodified{2013-03-22 17:24:13}
\pmowner{stitch}{17269}
\pmmodifier{stitch}{17269}
\pmtitle{properties of Bernstein polynomial}
\pmrecord{10}{39775}
\pmprivacy{1}
\pmauthor{stitch}{17269}
\pmtype{Derivation}
\pmcomment{trigger rebuild}
\pmclassification{msc}{65D17}

\endmetadata

% this is the default PlanetMath preamble.  as your knowledge
% of TeX increases, you will probably want to edit this, but
% it should be fine as is for beginners.

% almost certainly you want these
\usepackage{amssymb}
\usepackage{amsmath}
\usepackage{amsfonts}

% used for TeXing text within eps files
%\usepackage{psfrag}
% need this for including graphics (\includegraphics)
%\usepackage{graphicx}
% for neatly defining theorems and propositions
%\usepackage{amsthm}
% making logically defined graphics
%%%\usepackage{xypic}

% there are many more packages, add them here as you need them

% define commands here

\begin{document}
The \emph{Bernstein polynomials} $B_i^n(t)$ have the following properties:

\subsection{Non negativity}
The polynomials are non-negative over the interval $[0, 1]$.
$$B_i^n(t)\ge 0 \quad\quad 0\le t\le 1$$

\subsection{Symmetry}
The set of polynomials of degree $n$ is symmetric with respect to $t=1/2$.
$$B_i^n(t)=B_{n-i}^n(1-t)$$

\subsection{Maximum}
Each polynomial has only one maximum over the interval $[0,1]$ at $t=\frac{i}{n}$.

\subsection{Normalization}
The set of polynomials of degree $n$ forms a partition of unity.
$$\sum_{i=0}^{n} B_i^n(t)=1$$

\subsection{Degree raising}
A polynomial can always be written as a linear combination of polynomials of higher degree.
$$B_i^{n-1}(t)=\frac{n-i}{n}B_i^n(t)+\frac{i+1}{n}B_{i+1}^n(t)$$
%%%%%
%%%%%
\end{document}
