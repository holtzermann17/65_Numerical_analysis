\documentclass[12pt]{article}
\usepackage{pmmeta}
\pmcanonicalname{NewtonAndCotesFormulas}
\pmcreated{2013-03-22 14:50:28}
\pmmodified{2013-03-22 14:50:28}
\pmowner{drini}{3}
\pmmodifier{drini}{3}
\pmtitle{Newton and Cotes formulas}
\pmrecord{7}{36510}
\pmprivacy{1}
\pmauthor{drini}{3}
\pmtype{Definition}
\pmcomment{trigger rebuild}
\pmclassification{msc}{65D32}
\pmsynonym{Newton-Cotes}{NewtonAndCotesFormulas}
\pmrelated{SimpsonsRule}
\pmrelated{CodeForSimpsonsRule}

\usepackage{graphicx}
%%%\usepackage{xypic} 
\usepackage{bbm}
\newcommand{\Z}{\mathbbmss{Z}}
\newcommand{\C}{\mathbbmss{C}}
\newcommand{\R}{\mathbbmss{R}}
\newcommand{\Q}{\mathbbmss{Q}}
\newcommand{\mathbb}[1]{\mathbbmss{#1}}
\newcommand{\figura}[1]{\begin{center}\includegraphics{#1}\end{center}}
\newcommand{\figuraex}[2]{\begin{center}\includegraphics[#2]{#1}\end{center}}
\newtheorem{dfn}{Definition}
\begin{document}
The usual way of numerically integrate a function, is to find a simpler function which approximates the given function and then integrating the interpolation function.
That is, if we want to find $\int_a^b f(x)\,dx$, we find an approximating function $p(x)$ such that $f(x)$ and $p(x)$ be close (on some concept of distance) and then we say
\[
\int_a^b f(x)\,dx\approx \int_a^b p(x)\,dx
\]

The simplest approximation functions are polynomials. If we evaluate $f(x)$ at some points $x_0,x_1,\ldots,x_n$, we can use Lagrange's interpolating polynomial to find a polynomial $p(x)$ with degree $n$ such that $p(x_j)=f(x_j)$ for $j=0,1,\ldots,n$.

Newton and Cotes' integration formulas are obtained when the $x_0,x_1,\ldots,x_n$ are sampled evenly over the interval, and then Lagrange interpolating polynomials are used to approximate the function.


The Newton and Cotes formulas for small values of $n$ are given on the following table.
\begin{center}
\begin{tabular}{|c|c|c|}
\hline\hline
${n}$& ${\int p(x)}$& {\bf Name}\\
\hline
$1$ & $\frac{h}{2}(f(x_0) + f(x_1))$ & Trapezoidal rule\\
$2$ & $\frac{h}{3}(f(x_0) + 4f(x_1) + f(x_2))$& Simpson's rule\\
$3$ & $\frac{3h}{8}(f(x_0)+3f(x_1)+3f(x_3)+f(x_3))$& Simpson's 3/8 rule\\
$4$ & $ \frac{2h}{45}(7f(x_0)+32f(x_1)+12f(x_2)+32f(x_3)+7f(x_4))$& Milne's rule
\\
\hline\hline
\end{tabular}
\end{center}
recalling that $x_0,x_1,\ldots,x_n$ are evenly spaced on $[a,b]$.

Since the Simpson's rule is actually the Newton and Cotes formula for $n=2$, the proof of Simpson's rule illustrates this method.
%%%%%
%%%%%
\end{document}
