\documentclass[12pt]{article}
\usepackage{pmmeta}
\pmcanonicalname{ExampleOfGramSchmidtOrthogonalization}
\pmcreated{2013-03-22 15:03:02}
\pmmodified{2013-03-22 15:03:02}
\pmowner{drini}{3}
\pmmodifier{drini}{3}
\pmtitle{example of Gram-Schmidt orthogonalization}
\pmrecord{5}{36766}
\pmprivacy{1}
\pmauthor{drini}{3}
\pmtype{Example}
\pmcomment{trigger rebuild}
\pmclassification{msc}{65F25}
\pmrelated{ProofOfGramSchmidtOrthogonalizationProcedure}

\usepackage{graphicx}
%%%\usepackage{xypic} 
\usepackage{bbm}
\newcommand{\Z}{\mathbbmss{Z}}
\newcommand{\C}{\mathbbmss{C}}
\newcommand{\R}{\mathbbmss{R}}
\newcommand{\Q}{\mathbbmss{Q}}
\newcommand{\mathbb}[1]{\mathbbmss{#1}}
\newcommand{\figura}[1]{\begin{center}\includegraphics{#1}\end{center}}
\newcommand{\figuraex}[2]{\begin{center}\includegraphics[#2]{#1}\end{center}}
\newtheorem{dfn}{Definition}
\usepackage[fleqn]{amsmath}
\begin{document}
Let us work with the standard inner product on $\R^3$ (dot product) so we can get  a nice geometrical visualization.

Consider the three vectors
\begin{align*}
v_1&=(3,0,4)\\
v_2&=(-6,-4,1)\\
v_3&=(5,0,-3)
\end{align*}
which are linearly independent (the determinant of the matrix $A=(v_1|v_2|v_3)=116\neq 0)$ but are not orthogonal.
\figura{vectores1.eps}
We will now apply Gram-Schmidt to get three vectors $w_1,w_2,w_3$ which span the same subspace (in this case, all $R^3$) and orthogonal to each other.

First we take $w_1=v_1=(3,0,4)$. Now,
\[w_2= v_2 - \frac{w_1\cdot v_2}{\Vert w_1\Vert^2}w_1\]
that is,
\[
w_2 = (\frac{-108}{25},-4,\frac{81}{25})
\]
and finally
\[
w_3=v_3-\frac{w_1\cdot v_3}{\Vert w_1\Vert^2}w_1 - \frac{w_2\cdot v_3}{\Vert w_2\Vert^2}w_2
\]
which gives
\[
w_3=(\frac{1856}{1129},\frac{3132}{1129},\frac{1392}{1129})
\]
and so $\{w_1,w_2,w_3\}$ is an orthogonal set of vectors such that $\langle w_1,w_2,w_3\rangle=\langle v_1,v_2,v_3\rangle$. 
\figura{vectores2}


If we rather consider $\{w_1/\Vert w_1\Vert,w_2/\Vert w_2\Vert,w_3/\Vert w_3\Vert\}$ then we get an orthonormal set.
%%%%%
%%%%%
\end{document}
