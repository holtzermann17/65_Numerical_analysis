\documentclass[12pt]{article}
\usepackage{pmmeta}
\pmcanonicalname{IntegralTransform}
\pmcreated{2013-03-22 12:34:03}
\pmmodified{2013-03-22 12:34:03}
\pmowner{PrimeFan}{13766}
\pmmodifier{PrimeFan}{13766}
\pmtitle{integral transform}
\pmrecord{10}{32815}
\pmprivacy{1}
\pmauthor{PrimeFan}{13766}
\pmtype{Definition}
\pmcomment{trigger rebuild}
\pmclassification{msc}{65R10}
\pmrelated{ContourIntegral}
\pmrelated{GroupHomomorphism}
\pmdefines{kernel}
\pmdefines{transform parameter}

\usepackage{amssymb}
\usepackage{amsmath}
\usepackage{amsfonts}

\begin{document}
\PMlinkescapeword{generic}
\PMlinkescapeword{transform}

A generic \emph{integral transform} takes the form $$F(p)  =  \int_\alpha^\beta K(p, t) f(t)dt,$$ with $p$ being the {\em transform parameter}.

Note that the transform takes a function $f(t)$ and produces a new function $F(p)$.

The function $K(p, t)$ is called the \emph{kernel} of the transform.  The kernel of an integral transform, along with the \PMlinkname{limits}{DefiniteIntegral} $\alpha$ and $\beta$, distinguish a particular integral transform from another.  

\subsection*{Examples}
\begin{itemize}
\item Laplace transform
\begin{gather*}
\alpha = 0,\; \beta = \infty,\; K(p,t) = e^{-pt},\\
F(p) = \int\limits_0^\infty e^{-pt} f(t) dt.
\end{gather*}

\item Laplace-Carson transform
\begin{gather*}
\alpha = 0,\; \beta = \infty,\; K(p,t) = pe^{-pt},\\
F(p) = \int\limits_0^\infty pe^{-pt} f(t) dt.
\end{gather*}

\item Fourier transform
\begin{gather*}
\alpha = -\infty,\; \beta = \infty,\; K(p,t) = \frac{1}{\sqrt{2\pi}}e^{-ipt},\\
F(p) = \frac{1}{\sqrt{2\pi}}
	\int\limits_{-\infty}^\infty e^{-ipt} f(t) dt.
\end{gather*}
\end{itemize}
%%%%%
%%%%%
\end{document}
